\documentclass[12pt]{article}

\usepackage[margin=1in]{geometry}
\usepackage{amsmath, amssymb, amsfonts}
%\usepackage{natbib}
\usepackage{graphicx}
\usepackage{color} %% red, green, and blue (for screen display) and cyan, magenta, and yellow
\definecolor{Navy}{rgb}{0,0,0.8}
\usepackage{hyperref}
\hypersetup{colorlinks=true, urlcolor={Navy}, linkcolor={Navy}, citecolor={Navy}}

\parskip 7.2pt

\newcommand{\compresslist}{%
%\setlength{\itemsep}{1pt}%
\setlength{\itemsep}{0pt}%
\setlength{\parskip}{0pt}%
\setlength{\parsep}{0pt}%
}

\newcommand{\pb}{\mathbb{P}}
\newcommand{\E}{\mathbb{E}}
\newcommand{\V}{\mathbb{V}}
\newcommand{\C}{\mathbb{C}}
\newcommand{\bea}{\begin{align*}}
\newcommand{\eea}{\end{align*}}
\newcommand{\beq}{\begin{equation}}
\newcommand{\eeq}{\end{equation}}
\newcommand{\be}{\begin{enumerate}}
\newcommand{\ee}{\end{enumerate}}
\newcommand{\bi}{\begin{itemize}}
\newcommand{\ei}{\end{itemize}}
\renewcommand{\baselinestretch}{1}

\title{\texttt{SQUAREM}: Accelerating the Convergence of EM, MM and Other Fixed-Point Algorithms}
\author{Ravi Varadhan}
\date{}
\usepackage{Sweave}
\begin{document}

%\VignetteIndexEntry{SQUAREM Tutorial}
%\VignetteDepends{setRNG}
%\VignetteKeywords{EM algorithm, fixed-point iteration, acceleration, extrapolation}
%\VignettePackage{SQUAREM}

\maketitle

\section{Overview of SQUAREM}
''SQUAREM'' is a package intended for accelerating slowly-convergent contraction mappings.  It can be used for accelerating the convergence of slow, linearly convergent contraction mappings such as the EM (expectation-maximization) algorithm, MM (majorize and minimize) algorithm, and other nonlinear fixed-point iterations such as the power method for finding the dominant eigenvector.  It uses a novel approach callled squared extrapolation method (SQUAREM) that was proposed in Varadhan and Roland (Scandinavian Journal of Statistics, 35: 335-353), and also in Roland, Vardhan, and Frangakis (Numerical Algorithms, 44: 159-172).

The functions in this package are made available with:

\begin{Schunk}
\begin{Sinput}
> library("SQUAREM") 